\documentclass{beamer}

\title{Tic Tac Toe using Reinforcement Learning}
\author[Group 12]{Sanjana Chakravarty, Mahjabeen Azad, Shobhitaa Barik}
\date{June 2020}
\usetheme{Madrid}
\usecolortheme{seahorse}

\begin{document}

\begin{frame}
    \titlepage
\end{frame}

\begin{frame}
    \frametitle{Overview}
    Training a tic tac toe game with reinforcement learning to improve the AI’s success rate with experience.
\end{frame}

\begin{frame}
    \frametitle{Technology/Stack/Framework}
        \begin{itemize}
            \item<1-> Python and python libraries
        \end{itemize}
\end{frame}

\begin{frame}
    \frametitle{Description}
    \begin{block}{Reinforcement Learning}        
        \begin{itemize}
            \item<1-> Reinforcement learning is the training of machine learning models to make a sequence of decisions.
            \item<2-> The agent learns to achieve a goal in an uncertain, potentially complex environment.
            \item<3-> An artificial intelligence faces a game-like situation.
            \item<4-> The computer employs trial and error to come up with a solution to the problem.
            \item<5-> Its goal is to maximize the total reward.
        \end{itemize}
    \end{block}
\end{frame}

\begin{frame}
    \frametitle{Description}
          
     Rules of Tic Tac Toe:
    \begin{itemize} 
        \item The game is played on a grid that's 3 squares by 3 squares.
        \item Two people are required to play this game (in this case, a HUMAN and a COMPUTER). You are X, your friend (or the computer) is O. Players take turns putting their marks in empty squares.
        \item The first player to get 3 of her marks in a row (up, down, across, or diagonally) is the winner.
        \item When all 9 squares are full, the game is over. If no player has 3 marks in a row, the game ends in a tie.
    \end{itemize}
\end{frame}

\begin{frame}
    \frametitle{Status}
    \begin{itemize}
        \item Started working on 2 player tic tac toe game (which will be upgraded to a human vs computer game)
        \item Started reading about Reinforcement Learning
    \end{itemize}
\end{frame}

\begin{frame}
    \frametitle{Learning/Difficulties Faced}
\end{frame}

\end{document}
