\documentclass[14pt]{beamer}

\title[WTEF 2020]{Tic Tac Toe}
\subtitle{using Reinforcement Learning}
\author[Group 12]{Sanjana Chakravarty, Mahjabeen Azad, Shobhitaa Barik}
\date{June 2020}

\usetheme{Madrid}
\usecolortheme{crane}
\usepackage{textcomp}
\usepackage{xcolor}

\definecolor{myPink}{cmyk}{0, 0.7808, 0.4429, 0.1412}
\definecolor{myAmber}{rgb}{1.0, 0.49, 0.0}

\begin{document}

\begin{frame}
    \titlepage
\end{frame}

\begin{frame}{Overview}
    \begin{center}
        \textcolor{myAmber}{Player vs Computer Tic Tac Toe Game}
    \end{center}
   
    \begin{center}
    Training a tic tac toe game with reinforcement learning to improve the AI\textquotesingle s success rate with experience.
    \end{center}
\end{frame}

\begin{frame}{Technology Stack/Framework}
        \begin{itemize}
            \item Language: Python
            \item Pygame for graphical interface 
            \item ML library yet to be finalised (Tensorflow/Keras/OpenAI etc.)
        \end{itemize}
\end{frame}

\begin{frame}{Brief Description}
    \begin{block}{Reinforcement Learning}        
        \begin{itemize}
            \item<1-> Agent learns from interaction with unpredictable environment to achieve goal
            \item<2-> Selects action by trial and error
            \item<3-> If action is desirable, gets reward
            \item<4-> Goal: maximize the total reward
        \end{itemize}
    \end{block}
\end{frame}

\begin{frame}{Status}
    \begin{description}[STATUS]
        \item[\color{myPink}{24 June}] Coded a simple two-player Tic Tac Toe game
        \item[\color{myPink}{27 June}] Modified to player vs computer game
        \item[\color{myPink}{28 June}] Learnt about classification and neural networks
        \item[\color{myPink}{29 June}] Started reading about different RL algorithms
        \item[\color{myPink}{04 July}] Started looking into different libraries
    \end{description}
\end{frame}

\begin{frame}{Learning and Difficulties Faced}
    \begin{itemize}
        \item Difficulties faced
            \begin{itemize}
                \item The maths involved in the RL algorithms is difficult to understand
                \item Some of the algorithms are interlinked and it's difficult to make out which is more suitable for which situations
            \end{itemize}
        \item Learning
            \begin{itemize}
            \item Basics of classification and neural networks on the go
            \item Operating git in more detail
            \end{itemize}
    \end{itemize}
\end{frame}

\begin{frame}{Sneak peak into our ambitious side}
    \begin{center}
        Future Venture: \textcolor{myAmber}{Tigers} and \textcolor{gray}{Goats}
    \end{center}
\end{frame}

\end{document}
